% !TEX root = illustrator_submission.tex

\section{Introduction}
\label{sec:intro}

%% The ``\copyrightspace'' command must be the first command after the 
%% start of the first section of the body of your paper. It ensures the
%% copyright space is left at the bottom of the first column on the first
%% page of your paper.

\copyrightspace

Designers and artists worldwide rely on \AdobeIllustrator/
to design and edit
resolution-independent 2D artwork and typographic content.
\Illustrator/ was Adobe's very first application when released over 27 years ago.

Prior to our work, no version utilized graphics hardware to accelerate \Illustrator/'s rendering.
All rendering was performed entirely by the CPU.
This situation is in stark contrast to the now ubiquitous GPU-acceleration of 3D graphics rendering
in Computer-Aided Design, Animation, and Modeling applications.  So while other graphical content creation applications
readily benefit from the past 15 years of improvements in GPU functionality and performance, \Illustrator/
could neither benefit from nor scale with the tremendous strides in GPU functionality and performance.
Our work remedies this situation as Figure~\ref{fig:teaser} shows.

The starting point for our work is OpenGL 4.4 \cite{GL44spec}
and the GPU-accelerated ``stencil, then cover'' path rendering functionality described in \cite{KilgardBolz2012}.
While the \NVpr/ OpenGL extension \cite{NVpr}
provides very fast and resolution-independent rendering of first-class path objects just as we need, \Illustrator/'s
rendering model requires much more than merely rendering paths.  
We had to develop our own strategies to handle features of
\Illustrator/ that, while dependent on path rendering, require considerably more sophisticated orchestration of the GPU.
We focus on the GPU-based techniques we developed and productized to accomplish
full GPU-acceleration of \Illustrator/.

Adobe
originally developed \Illustrator/ as a vector graphics editor for PostScript \cite{PLRM} which implements the
imaging model described by \cite{Warnock:1982:DIG:800064.801297}.
\Illustrator/ today depends on the Portable Document Format (PDF) standard for its underlying rendering capabilities
as described by the ISO 32000 standard \cite{PDF-Spec}.  The evolution of PostScript to PDF introduced a number
of sophisticated graphics capabilities for dealing with printer color spaces, compositing, and photorealistic artistic shading.
These capabilities developed in parallel with but isolated from
contemporaneous hardware-oriented improvements in interactive 3D graphics.
PDF and \Illustrator/ incorporated features and solutions relevant to
the print industry such as subtractive color spaces, typography, and
rasterization of 2D spline-based content targeting print engines with
extremely high pixel density.  However GPU hardware designers essentially
ignored such concerns---instead focusing on interactive 3D graphics.

For example, GPUs specialize at rendering RGB colors---typically with alpha, so RGBA---for
display on emissive RGB monitors.  Indeed the framebuffer, texture, and memory subsystems of GPUs are 
tailored specifically for handling 4-component RGBA colors.
However \Illustrator/ users target printed color output and expect accurate color reproduction
so the default document color mode in \Illustrator/ is CMYK, a color space intended for printed content.
\Illustrator/ also does support RGB documents but CMYK matters more to professional users.

Rather than just 4 color components as GPUs are designed to handle well,
CMYK involves at least 5 components (the process ink colors {\em cyan}, {\em magenta},
{\em yellow}, and {\em black} + alpha) and then additional {\em spot} colors assigned to specific custom inks.
Simply converting CMYK colors to RGB and rendering in an RGB color space is not the same as rendering into
a proper CMYK color space as Figure~\ref{fig:cmyk-vs-rgb-emulation} shows.
Among the differences, CMYK is a subtractive color space while RGB is an additive color
space so color arithmetic such as blending must account for this difference.
When a GPU's data paths are so specialized for processing RGBA values with exactly
4 components, handling 5 or more components in a manner consistent with PDF
requirements and at reasonable efficiency requires novel treatment.

\subsection{Motivation}

Our work to GPU-accelerate \Illustrator/ has the obvious goal of improving
the rendering speed and interactivity to improve user productivity.  \Illustrator/
documents can become extremely complex and so poor rendering performance
can frustrate a designer's creativity.

We also considered display technology trends, particularly increasing screen resolutions and
pixel densities.  So-called {\em 4K} resolutions such as Ultra HD at 3840x2160
are particularly interesting to us.
CPU performance scaling for rendering in \Illustrator/ has clearly not
been adequate to keep up with the increasing number of pixels needing to be rendered.
Recent 5K Ultra HD (5120x2880) displays make this more pressing.

%%%We also observe how the proliferation of non-PC devices are creating a multitude of pixel resolutions
%%%to which to deploy resolution-independent 2D content.

While our focus has been fully accelerating {\em existing} features of the PDF rendering model, we
anticipate our transition of \Illustrator/'s rendering to the GPU allows us to accelerate graphics
operations, known as {\em effects} in \Illustrator/, that are handled ``above'' the PDF rendering model currently.
Gaussian blurs and image warps are obvious examples of effects the GPU could significantly accelerate.
\ifdefined\NOSHOW
Prior to \Illustrator/'s GPU-acceleration, processing effects with the
GPU was simply untenable because the inputs and outputs of effects reside in system
memory where they are not efficiently accessible to the GPU.
\fi

% When all Illustrator rendering occurs on the GPU, effect acceleration becomes possible.

\subsection{Contributions and Outline}
\label{sec:contributions}

Our key contributions are:
\begin{itemize}

\item Novel repurposing of the GPU's multiple RGBA render buffers for {\em NChannel} color space rendering of CMYK process colors and additional spot colors with full blending support.

\item GPU algorithms to composite both isolated and non-isolated transparency groups properly.

\item Tessellating gradient meshes to shade paths via GPU hardware tessellation.

\item Harnessing ``stencil, then cover'' path rendering to support arbitrary path clipping, pattern shading, and knockout groups.

\item Rendering complex vector scenes with the GPU many times faster than comparable multi-core CPU rendering.

\end{itemize}
These contributions are broadly applicable beyond \Illustrator/ and benefit
any GPU-accelerated software system that
utilizes PDF, SVG, or similar printing, web, or vector graphics standards.
We anticipate web browsers, other 2D digital content creation applications,
document previewers, and even printers will use our contributions
to accelerate existing standards for resolution-independent 2D vector graphics.

Furthermore we expect GPU-acceleration of \Illustrator/ to motivate graphics hardware architects
to focus on specific optimizations for this important rendering model.  Likewise research in GPU-based  techniques for vector
graphics becomes substantially easier to productize.

Section~\ref{sec:relatedwork} outlines relevant prior work.  Section~\ref{sec:background} provides useful background for \Illustrator/'s
software architecture relevant to rendering. 
Section~\ref{sec:blendmodes} discusses GPU support for blend modes, a prerequisite for our contributions.
Sections~\ref{sec:cmyk} to \ref{sec:shading} describe the primary techniques we developed
to GPU-accelerate Illustrator fully.  Section~\ref{sec:comparison} compares and contrasts our GPU-acceleration to Illustrator's existing CPU renderer.   Section~\ref{sec:performance} presents our performance improvements.  Section~\ref{sec:conclusion} concludes
with a call for standardization and discussion of future plans.

\subsection{Status}

The June 2014 release of \AdobeIllustratorCC/ first incorporated our
GPU-acceleration efforts.  Adobe reports over 3.4 million Creative
Cloud subscribers as of December 2014.  A substantial fraction of
these subscribers use \Illustrator/.
The complete
contributions we discuss are refinements beyond the initial release.
In particular, the 2015 version introduces CMYK
(Section~\ref{sec:cmyk}) and tessellation-based gradient meshes (Section
\ref{sec:gradient-meshes}).
