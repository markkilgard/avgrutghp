% !TEX root = illustrator_submission.tex

\section{Conclusions}
\label{sec:conclusion}

We have succeeded in GPU-accelerating \Illustrator/ despite
decades of being unadvantaged by graphics hardware.  Our
benchmarking shows significant speed-ups, particularly at 4K resolution.
Our contributions introduce novel techniques for supporting CMYK color
space rendering on existing GPUs, proper transparency group support
including non-isolated groups and knock-out, and mapping PDF's gradient
mesh shading to GPU tessellation.

\subsection{Broader Hardware Support}
\label{sec:standardization}

OpenGL 4.4 is the multi-vendor standard baseline for our GPU-acceleration
effort but \Illustrator/ also requires the \NVpr/ and \KHRbea/
extensions.  As a practical matter, today just NVIDIA GPUs (Fermi
generation \cite{Fermi} and beyond) on Windows support the prerequisite
OpenGL functionality.  \Illustrator/ supports a wide range
of system configurations so we
naturally want these GPU-acceleration benefits supported more broadly.
We are exploring ways to support a broader range of GPUs.
Standardization of \NVpr/ would make this much easier.

\ifdefined\NOSHOW
We happily note all the blend mode functionality we need
is now standardized as \KHRbea/ \cite{KHRbeaSpec} though other shader-based
alternatives (see Section \ref{sec:blendalternatives}) to implement blend modes are worth investigating.
\fi

\subsection{Future Work}
\label{sec:futurework}

Much of the user interface of \Illustrator/ today assumes
re-rendering the scene is slow and expensive.  For example, the
user interface encourages users to {\em isolate} a portion of the
scene to avoid rendering the complete scene.  Similarly editing
often happens by dragging overlaid ``blue lines'' rather than a more
{\em what-you-see-is-what-you-get} interaction model.  We hope the
GPU-acceleration we have introduced into \Illustrator/ will facilitate
more powerful, intuitive, and fluid user interfaces for vector graphics
editing.

We note a number of inspired research efforts relevant to vector
graphics editing that are either conceived with GPU-acceleration
in mind---such as diffusion curves \cite{Andronikos2013}---or
introduce vector graphics complexity that overwhelms conventional
CPU-based vector graphics rendering---such as digital micrography
\cite{Maharik:2011:DM:2010324.1964995}.  We hope by bringing
\Illustrator/ to the GPU, these techniques will become tractable to
support within \Illustrator/ and thereby be adopted by digital artists.

Our framebuffer memory usage is substantial.  We want to incorporate
NVIDIA's \NVfms/ OpenGL
extension \cite{FramebufferMixedSamples}
that allows an OpenGL framebuffer object to have fewer color
samples than stencil samples (and no depth samples) to reduce memory usage without reducing
the rasterization quality of path rendering.  

We believe graphics hardware architects can improve the support for
print-oriented features---in particular the CMYK color space.  Doing so
would greatly reduce the memory bandwidth, memory footprint, and greatly
simplify the complex orchestration of multiple RGBA color buffers required
to accomplish CMYK rendering.

\ifreviewelse
{}
{
\section*{Acknowledgments}

We thank:
David Aronson,
Rui Bastros,
Jeff Bolz,
Rajesh Budhiraja,
Nathan Carr,
Qingqing Deng,
Vineet Punjabi,
E Ramalingam,
Anubhav Rohatgi,
Lekhraj Sharma,
Gopinath Srinivasan,
Tarun Beri,
and our anonymous reviewers.
}
