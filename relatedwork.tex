% !TEX root = illustrator_submission.tex

\section{Related Work}
\label{sec:relatedwork}

The PDF specification \cite{PDF-Spec} describes in detail the rendering model \Illustrator/ implements.
Kilgard and Bolz \shortcite{KilgardBolz2012} provides a good review of GPU-acceleration approaches for path rendering
and is the crucial underlying functionality upon which our work relies.  

Our interest in rendering complex
vector graphics scenes is similar to the random-access vector texture approach of
\cite{Nehab:2008:RRG:1457515.1409088} and its refinement by \cite{Leben:Thesis:2010}.
However \Illustrator/'s raison d'\^{e}tre is arbitrary editing of vector graphics scenes in a very general
setting (permitting CMYK color spaces, blend modes, transparency groups, etc.) so the need to
re-encode the scene's vector texture whenever the scene is manipulated is unappealing.

Vector textures are also unappealing because we must support all of \Illustrator/'s rich feature set but
a vector texture scheme 
effectively centralizes support for the scene's entire rendering requirements.  This means
the fragment shader used to decode the vector texture approach
must be prepared to incorporate every possible vector graphics feature in the scene.
The advantage of random-access decoding when rendering from a vector texture, while interesting, is not particularly relevant
to \Illustrator/.

Recent work by \cite{GanEtAl14} has significantly advanced the vector texture approach by using CUDA
to implement massively-parallel vector graphics on the GPU
rather than the conventional graphics pipeline.
Section~\ref{sec:mpvg} compares their reported performance to ours so we defer more discussion until then.

\subsection{Alternative Vector Graphics Editors}

While there are many alternative path-based vector graphics editors, such as Inkscape \cite{kirsanov2009inkscape}
and CorelDRAW \cite{CorelDRAWbook},
no existing vector graphics editor significantly exploits graphics hardware
acceleration to the best of our knowledge.

While not a conventional path-based editor,
the Mischief application \cite{MischiefWebPage} for scalable drawing, sketching, and painting is GPU-accelerated.
Mischief takes an Adaptive Distance Field (ADF) approach
\cite{Frisken:2006:DDF:1185657.1185675}.  ADFs represent shape implicitly
rather than the explicit path-based approach used by \Illustrator/ and PDF
to represent shape. A hybridization of Mischief's implicit approach and our explicit approach may be possible
given the commonality of GPU-acceleration.

\subsection{Smooth Shading} 

The PDF standard provides several shading operators for continuous color. 
Linear and radial gradients are very familiar to 2D digital artists and are a standard way for
integrating continuous color into vector graphics content.  For example, the SVG standard \cite{SVG-Spec} supports
both linear and radial gradients.  Our GPU-acceleration efforts accelerate both linear and radial
gradients using straightforward cover shaders as described in \cite{KilgardBolz2012}.

\Illustrator/'s {\em gradient mesh} tool provides a way to shade paths using
a mesh of Coons patches to assign and edit continuous color within a path \cite{PDF-Spec,CoonsPaper}.
Skilled artists use gradient meshes to create editable photorealistic resolution-independent artwork,
often sampling colors for the mesh control points from photographs.
%Researchers have explored techniques to assist in the generation of
%gradient meshes \cite{Sun:2007:IVU:1276377.1276391,Lai:2009:ATG:1531326.1531391}.

A different approach to
assigning smooth color to vector graphics artwork called {\em diffusion curves} \cite{Orzan:2013:DCV:2483852.2483873,Sun:2012:DCT:2185520.2185570,Ilbery:2013:BDC:2508363.2508426} relies on
partitioning 2D space with vector-based primitives and letting color naturally ``diffuse'' through the
scene to a steady-state that respects the partitions.  The color diffusion process is computationally intensive and
well-suited for the massively parallel nature of the GPU.  Our immediate interest is accelerating
\Illustrator/'s existing gradient mesh functionality though we anticipate our transition to the GPU
facilitates the adoption of more intuitive gradient methods otherwise too expensive for the CPU.
